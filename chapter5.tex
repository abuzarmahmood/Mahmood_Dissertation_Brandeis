\begin{refsection}

\chapter{Discussion}

While it is widely recognized that processing of sensory information in the brain is governed by dynamics at multiple timescales, these dynamics have only recently begun to be explored in the context of taste processing. Investigation into the role of multi-region interactions follows a similar trend. While it has been shown that brain regions “up and down” the taste circuit (including the pons, lateral hypothalamus, basolateral amygdala, gustatory cortex, and medial prefrontal cortex) all respond to taste in a dynamic fashion (\cite{katz2001a,fontanini2009a,jezzini2013a,li2013a,baez-santiago2016a}), the work done in this thesis represents the first attempt (to the best of our knowledge) to investigate and characterize interactions between taste regions during processing.

In Chapter 2, we investigate how long-term dynamics on the order of minutes and hours – in the form of sickness caused by gastric malaise- affect taste processing in the gustatory cortex. Despite having only a transient impact on GC LFP spectral properties, sickness induces a long-lasting change in its evoked neural response. It is surprising that there is not persistent "signal” for sickness in GC and it is unlikely that such a signal is not present anywhere in the brain – please note however, that this does not mean that it is the brain that is maintaining such a signal, and such a signal could simply be present in the form of altered spontaneous activity due to input from the viscera. this observation further underpins the idea that GC is not the focus of processing in the taste circuit and that other brain regions are simultaneously playing significant roles. We further see that GC palatability processing is pushed towards a binary choice under sickness rather than a spectrum of hedonic values under normal processing. One may speculate that this increased similarity to BLA’s palatability encoding likely has something to do with changes in BLA-GC functional connectivity during sickness. These points are further discussed in Future Directions section below.

In Chapter 3, we investigate changes in neural dynamics on the order of hundreds of milliseconds and seconds, as well as exploring communication between brain regions involved in taste processing. This investigation tests the long-held belief that the similarity in BLA and GC taste evoked dynamics is due to a direct interaction between them. First we test that BLA population activity contains similar state transitions to GC, which not only suggests that BLA is a similar dynamical system to GC, but also that the two regions could be considered a single system. We then go on to test this assertion by investigation the coordination between BLA and GC. We show that BLA and GC responses are linked at the whole-trial (spike-count correlations), epoch (LFP phase coherence), and state-transition levels, we clearly see that this is not due to a “default” connection between the two regions due to their strong reciprocal connectivity, and that the BLA-GC interaction is dynamic (only appearing significantly around the late [palatability in GC] epoch), which suggests that BLA plays a significant role only later in the evoked taste response. This evidence is further corroborated by previous causal studies (\cite{lin2021a}) showing that while perturbation of the BLA$\rightarrow$GC pathway causes the largest impact on GC neural responses in the 250ms immediately following taste delivery, changes in GC coding are only visible during the palatability epoch. This can likely be explained by the fact that physical disruption of input by the perturbation will cause the largest change when the neural activity is also the highest (it has consistently been shown that that the neural response magnitude is highest immediately following taste delivery [i.e. during the early response epoch] and tapers off over the next 2 seconds), however, the role this projection plays is disturbed specifically during the palatability epoch.

Finally, in Chapter 4, we paint a picture for how neural activity is inherently a dynamic entity, such that the response of a neuron or brain region is ill-characterized without temporal context. This chapter provides a consilience for the previous chapters, discussing how neural responses are influenced by multiple physiological processes and external influences (including sensory input) at multiple timescales. Hence, the same neuron will subserve different computations at different times, and in corollary, will have different network interactions over time as well. Hence, the response of GC neurons changes across trials as governed by the onset of sickness, and zooming into a single trial, we see GC epochal/state dynamics governed by its interactions with BLA (and likely other brain regions as well).

The studies mentioned above, among much other extant research, highlight the importance of investigating how temporal dynamics govern processing in the taste circuit. Given this framework, there are many avenues of research that could be explored to further understand the interregional dynamics that contribute to taste processing and develop a better understanding of the “sources” of neural dynamics at different timescales. An integration of these results into the existing literature and future directions for the work presented in this thesis are reviewed in the section below.

\section{Future Directions}
As mentioned throughout this thesis, a clearer understanding of network interactions in taste processing is still being developed. While pharmacological inhibition and ablation studies have provided important clues to the necessity of brain regions in taste-related learning and normal taste processing, such extreme perturbations can be misleading and not provide a direct test of how the intact taste circuit normally carries out processing. An illustrative example for this case is the fact that rats can still gape despite removal of cortex, which would suggest that cortex plays no role in the generation of gapes. While that is true at face value, and the central pattern generator (CPG) responsible for gaping is certainly located in the brainstem, such an extreme perturbation misses the nuance that despite not being responsible for generating gapes, GC is causally involved in the initiation of the gaping “program” and hence exerts a significant modulatory influence on the CPG. Similarly, we can expect much nuance in the role played by the multitudinous connections within the brain in taste processing.

My suggested avenues for further investigation may be broadly divided into two categories, corresponding to further investigations into the influence of sickness on the taste circuit, and further investigation of network interactions in the taste circuit during taste processing. Additionally, it has been shown that modelling of neural activity across timescales (a.k.a. scale-free modelling) improves prediction accuracy for sensorimotor tasks (\cite{samek2016a,abbaspourazad2021a}); hence, while a direct investigation of ways to scale the gap between the short on long timescales mentioned here will certainly be indispensable for furthering our understanding of processing in the taste circuit, pursuit of this avenue is left for the future.

\subsection{Illness and taste processing}

\smallskip
\noindent \underline{Increased behavioral binarization of food preference under sickness} This would be a confirmatory experiment to test that the increased polarization/binarization of hedonic values as seen in GC under sickness. A simple (possibly naïve) method to carry this out would be to have rats lick to the same tastants given in the experiments in Chapter 2 and undergoing the same experimental conditions (Saline vs LiCl) while in a Brief Access Task rig. Increased similarity (while accounting for any potential floor effects caused by the animal’s illness) in tastants with similar hedonic values would confirm that this is not simply a neural idiosyncracy.

\smallskip
\noindent \underline{Investigating the role of posterior GC in sickness} It is known that posterior gustatory cortex receives more visceral projections and hence may contain stronger signals related to illness, however, differences between anterior and posterior GC have hitherto not been studied. While it seems unlikely that responses in posterior GC would be categorically different, if the influence of illness on neural activity is stronger in posterior GC, it may prove to a better site for further investigation of the influence of illness on taste processing.

\subsection{Functional connectivity dynamics in the taste circuit}
As mentioned above, there has been much investigation of the role of different regions in taste processing and taste-related learning using ablation and bulk pharmacological inhibition. However, the field has only recently begun moving away from these extreme perturbations and begun to investigate the role of connections between regions using targeted inhibition of projection neurons, or (better yet) optogenetic perturbation of axonal projections. However, the realm of “functional connectivity” is still a great unknown in taste processing, where the work presented in Chapter 3 is the first attempt to perform such a characterization, and the studies presented here can be easily extended into many multi-region directions.

\smallskip
\noindent \underline{Changes in other taste regions and in functional connectivity due to illness} While the work in Chapter 2 shows that GC activity is (understandably) altered under the state of illness, the same characterization has yet to be performed for other taste processing regions i.e., what influence does sickness have on activity in BLA or the lateral hypothalamus. Since palatability encoding in BLA is already fairly binary (good vs. bad), does that change (perhaps becoming even more polarized)? One may speculate that the loss of gradation in the GC response under sickness could be due to an increase in connectivity between GC and BLA, such that more nuanced information about tastes from other regions being communicated to GC is mitigated. One may further speculate that since we only observe a transient change in GC LFP during the onset of sickness, but no ongoing change is observed while the animal is sick, that the influence of sickness primarily “resides” in other brain regions (posterior GC, BLA, LH etc) and we are only observing the influence of sickness in GC due to communication of GC with these brain regions. In very liberal wording, the “epicenter” of sickness is a different brain region, yet we observe it’s influence on GC. One can also reasonably expect the communication pathways and dynamics between regions in the taste circuit to be altered due to illness.

\smallskip
\noindent \underline{Ascertaining “top-down” influences exerted by GC} While work in Chapter 3 shows that GC and BLA show coordinated activity, the extent of this evidence is phenomenological. Both GC and BLA almost certainly interact with other brain regions in a dynamic fashion, and it is certainly plausible that the tight state-transition level coordination we see in this study is a general occurrence in multiple regions throughout the taste circuit (i.e., multiple regions in the circuit transition together). Hence, it would be illuminating to causally test the role played by the GCBLA feedback projection. As a first pass, if our hypothesis is correct, we will see that by perturbing this projection we will not only see changes in BLA activity but, since we expect the dyad to be working as a loop, we will also observe an impact on activity in GC. One potential role of this projection could be to communicate “Identity” information (information about taste quality) from GC to BLA which the two regions then collectively transform into palatability. This claim stems from previous evidence showing the pharmacological inhibition of taste thalamus strongly diminishes taste-selectivity (discrimination) in GC starting in the middle epoch (identity epoch: approx. 250-750ms post-stim) in GC, while similar pharmacological inhibition of BLA only has a strong effect on reduction of palatability information in GC during the late epoch (750-1250ms post-stim). This suggests that the thalamic taste pathway supports the first “leg” of the evoked taste response, and this activity then engages the limbic pathway (either through GC or via direct thalamo-amygdalar projections). However, this speculation is largely unconstrained as we know little about the when-and¬-where of coordination between these brain regions, a challenge addressed by the following investigation.

\smallskip
\noindent \underline{Investigating circuit interactions dynamics in passive taste processing} LFP recordings have been previously used to probe the simultaneous evolution of activity in multiple brain regions, and methodologies such as granger causality and vector autoregression have been employed to estimate the magnitude and directionality of influence between these brain regions. However, in many cases, this influence is assumed to be static and long durations of neural activity are homogenously analyzed to assess such connectivity. Certainly, in our case, such a simplification would not be possible, hence more advanced techniques such as auto-regressive Hidden Markov Models or switching Linear Dynamical Systems will need to be employed to estimate state-specific interactions. A suggested set of regions to investigate would include GC, BLA, LH, and Thalamus, which may help further clarify the strength and durations of thalamic vs limbic contributions to taste processing. Such a study may potentially help cast the spotlight on previously unnoticed interactions in the taste circuit and suggest pathways for future causal investigation, and may help to elucidate any isolable roles different regions play in processing.

\section{Conclusion}
As mentioned in the introduction of this thesis, the computational power of neural networks (and hence the brain) is derived from non-linear transformations and long-scale interactions. While we have continued to study the non-linear progressions in taste processing, the long-scale interactions have only just begun to come into focus. The experiments noted above in no way diminish the importance of single-region investigation and analysis, but suggest that similar to hierarchies in temporal timescales observed in the brain, adding multi-region analysis into the hierarchy of taste research will allow a more fruitful interaction between the information gleaned from single-region and multi-region studies.


\printbibliography[title={References}]
\end{refsection}