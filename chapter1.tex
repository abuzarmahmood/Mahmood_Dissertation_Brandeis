\begin{refsection}

\chapter{Introduction}

Historically, neurons have been assumed to maintain simple, constant response profiles to stimuli. While neural responses were known to be modulated by the magnitude/intensity of the stimulus, neurons were considered transducers of stimuli such that there was a neural response when the stimulus is present, but not otherwise. Examples of this can be seen in investigations of taste processing in which a taste either evokes an on response in a single neuron or doesn’t (\cite{mark1988a,yaxley1988a,yamamoto1989a}), and in in investigation of higher order visual regions in which, again, a visual stimulus is considered to evoke a binary response (\cite{hasselmo1989a,gross1992a,rolls1995a}). However, an increasingly large number of studies demonstrate that neural responses are dynamic – which is unsurprising considering that the stimuli we interact with are all dynamic – and that these dynamics play an important role in neural processing (\cite{sugase1999a,katz2001a,brincat2006a,sadacca2016a,brincat2018a,saravani2019a}).

The work presented in this thesis explores how changes in neural processing are related to dynamics. The two data chapters attack this broad topic at (arguably) the two ends of the time-scale spectrum: 1) dynamics occurring on the order of hundreds of milliseconds; and 2) those occurring on the order of tens of minutes and hours, influence taste processing. In the final chapter, we discuss how these dynamics are intricately tied to the function of the neuron, such that the instantaneous function of the neuron (or neural population) is governed by when we are viewing that response in relation to long- and short-term dynamics.

As a prelude to these investigations, I will discuss briefly (since there is plenty to read in the main body of the thesis) what phenomena/actions processing at the above-mentioned timescales are associated with. I will go on to discuss how dynamics at these timescales can be jointly studied despite having disparate mechanisms and, finally, how at each time-scale these dynamics are a sign of multiple brain regions interacting with one another – something which has not been investigated in taste processing previously.


\section{Neural responses exhibit dynamics both slow and fast timescales}
Similar to any system capable of performing “complex” tasks (\cite{garcia1993a,cash2006a,tao2009a,hekstra2012a,frentz2015a,leong2016a,ribeiro2021a,graumann2022a,jonas2017a}), evoked activity in the brain exhibits dynamics at multiple spatio-temporal scales (\cite{engel2013a,tomasi2017a,kaplan2020a,raut2020a}). Such multi-scale organization is theorized to be directly linked to the information-processing capacity of the system (\cite{tononi2016a,mediano2022a}).  Neural dynamics on the order of hundreds of milliseconds are often seen in sensory processing and decision-making paradigms and are associated with the finest granularity of behavior (\cite{guo2014a,guo2015a,markowitz2018a,kaplan2020a,rastogi2020a,vincis2020a}), e.g. the process of delivering a single spurt of liquid into the mouth of a rat evokes multiple phases of mouth movements and concomitant (although not necessarily generated by the movement itself) phases of neural activity – hence, these neural dynamics are associated with the sensory processing required for a single ingestion/egestion behavior (\cite{sadacca2016a,mukherjee2019a}). Increasing the timescale, we see that body states (such as sickness or disengagement) are related to neural changes which occur on the timescale of minutes, tens of minutes, and hours.

While short-scale dynamics (up to seconds) are constrained by how long it takes activity to flow through the network (synaptic propagation and ion channel time-constants), as well as emergent phenomena which influence dynamics on timescales longer than these physical constraints (such as attractors formed by recurrence of activity), long-term dynamics are usually thought to be wrought by synaptic plasticity, in collaboration with the influence of neuromodulators (\cite{abarbanel2001a,abarbanel2002a,yasumatsu2008a,odonnell2011a,bazzari2019a,deperrois2020a}), which are further (likely jointly) governed by physiological rhythms/state and visceral input (\cite{critchley1996a,critchley2015a,ly2016a,azzalini2019a,paul2020a,candia-rivera2022a}). However, while the mechanisms generating these dynamics may be different, it is important to recognize that the neural dynamics at these different timescales themselves may be considered as parametric counterparts for their evoked effects and their role in behavioral and information processing. In essence, the effect of both short- and long-term mechanisms is to change the state of the system; hence, we can similarly study the influence of this “state” regardless of how long the state persists. However, things may certainly become more involved (and interesting) when investigating longer timescales as, depending on the granularity of the question, one may have to study the influence of nested dynamics in the system (as in Chapter 2).

\section{Regional communication dynamics (likely) underlie taste processing}
Our lived experience almost exclusively involves dynamic input (none of our sensory inputs are naturally point processes), and such dynamic input is better processed using recurrent - rather than feedforward - networks (\cite{kietzmann2019a,alamia2020a}). Hence, it is not surprising that our brains contain strongly recurrent connectivity (\cite{rigotti2010a,bergen2020a,matsumoto2022a}). 
Despite this fact, much previous work has been devoted to studying the feedforward aspects of processing in the brain (\cite{carleton2010a,heidari-gorji2021a}), a view that strongly persists today; however, the monopoly of this aspect of processing in neural processing is now being challenged, and more research is recognizing the degree to which such recurrent connectivity plays an important role. In the brain, we find recurrent connections not only within single brain regions (a circuit modulating itself), but also between multiple brain regions (modulating each other in a loop; \cite{mante2013a,hart2020a,kotekal2020a}).
The role and ubiquity of recurrent connectivity, coupled with the fact that many taste-processing regions exhibit “connected” dynamics on a single-neuron level (\cite{grossman2008a,fontanini2009a,jezzini2013a,li2013a,baez-santiago2016a}) strongly suggests that recurrent interactions between these regions underlie the dynamics that we observe. However, hitherto, there have been no experiments investigating "real-time” communication between regions processing taste. Chapter 3 provides the first such study investigating interactions in the taste circuit, and opens the door for extensions to study other taste-related processing (e.g. sickness, in Chapter 2) from a lens of changes in network interactions. Further avenues of investigation of communication in taste processing are mentioned in the thesis discussion.

\section{Summary}
In the following articles, this thesis will explore modulation of gustatory processing on long timescales (tens of minutes) due to change in body states (sickness), short timescale (hundreds of milliseconds) dynamics and coherence between brain regions for the purpose of taste processing while performing a comparison of information gleaned using “canonical” and new methodologies to assess communication (which will suggest that processing in the taste circuit is not strictly hierarchical, as previously thought), and conclude with a discussion that cautions neuroscientists to not assume the function of neurons without considering the temporal dynamics and context in which the response of that neuron is embedded.

\printbibliography[title={References}]
\end{refsection}